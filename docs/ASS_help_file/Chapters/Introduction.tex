\chapter{Introduction}
\label{cha:Introduction}

%\lipsum[1-4]

This open-source Python application is designed to simplify the analysis of spectroscopic data, with a particular focus on multi-peak fitting, batch data processing, and spectroscopy mapping (1D and 2D). Originally developed for academic and research use, the app provides a powerful, user-friendly graphical interface built with Tkinter, enabling scientists and students to interactively explore and interpret spectra without writing code. So far, this app can work with data acquired from HORIBA and Witec spectroscopes, but it can also define your own data file structure with the "User format" setting.

The application supports:
\begin{itemize}
    \item Importing and visualizing Raman spectra, including 1D and 2D spatial maps (e.g. line scans, hyperspectral images, etc.)
    \item Spectrum processing features such as minimum or linear subtraction, data filtering, or spectrum subtraction
.\item Flexible spectral fitting using standard peak models (Gaussian, Lorentzian, Voigt, etc.) with bounds and constraints
    \item Exporting and importing defined models significantly reduces the time required to recreate fitting models from scratch for new datasets or similar spectral analyzes.
    \item Batch fitting routines for consistent analysis of multiple spectra
    \item Export options for peak parameters, fitted curves, and processed datasets
    \item Analysis of the Raman map in 2D (spatial) and 1D (time scan, line scan, and electrochemical potential) modes
\end{itemize}

Whether you're working on a few spectra or hundreds of Raman maps, this app aims to accelerate insights, reduce manual labor, and improve reproducibility in spectroscopic analysis.