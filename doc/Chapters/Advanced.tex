\chapter{Advanced}
\label{cha:Advanced}
This section of the help document outlines the app's advanced features. These functionalities are all available through the "Advanced" option in the top menu.
\section{Filtering}
To reduce noise and enhance the interpretability of spectral data, the application offers several filtering methods. Each filter has its own characteristics and is suitable for different types of data and noise profiles. For application of the filter to the data just select the type of the filter you would like to apply, fitt the input parameters and click "Apply", this filter will be automatically applied to all data plotted in the Main window. To deactivate the filter, "Disable" button need to be clicked.

\subsection{Savitzky–Golay Filter}

The Savitzky–Golay filter smooths data by fitting a low-degree polynomial to a moving window of points using least squares. Unlike a simple moving average, it preserves the original shape and features of the signal, such as peak height and width.

\textbf{Parameters:}
\begin{itemize}
    \item \texttt{window length} – Number of points in the smoothing window. Must be odd and larger than the polynomial order.
    \item \texttt{polyorder} – Order of the polynomial used for fitting (e.g., 2 for quadratic, 3 for cubic).
\end{itemize}

\textbf{Pros:}
\begin{itemize}
    \item Retains peak shapes and derivatives.
    \item Effective for reducing random noise without distorting signal structure.
\end{itemize}

\textbf{Cons:}
\begin{itemize}
    \item Sensitive to window size; too large a window can flatten peaks.
    \item May introduce artifacts at the edges.
\end{itemize}

\textbf{Recommended Use:}  
Ideal for general-purpose smoothing when signal shape must be preserved (e.g., Raman or IR spectra).

\textbf{Avoid If:}  
The signal has sharp spikes or highly irregular noise; consider median filtering instead.

\subsection{Moving Average Filter}

The moving average (boxcar) filter replaces each point with the mean of its surrounding neighbors, effectively smoothing high-frequency noise.

\textbf{Parameters:}
\begin{itemize}
    \item \texttt{window\_size} – Width of the averaging window in number of points.
\end{itemize}

\textbf{Pros:}
\begin{itemize}
    \item Simple and fast to compute.
    \item Smooths out random fluctuations effectively.
\end{itemize}

\textbf{Cons:}
\begin{itemize}
    \item Can blunt sharp features and reduce resolution.
    \item Distorts peak height and width.
\end{itemize}

\textbf{Recommended Use:}  
Suitable for baseline smoothing or reducing random noise when precise peak shape is not critical.

\textbf{Avoid If:}  
The data contains closely spaced or narrow peaks that must be preserved accurately.

\subsection{Median Filter}

The median filter replaces each point with the median of the surrounding window, making it very effective at removing sharp outliers (e.g., cosmic rays or spikes).

\textbf{Parameters:}
\begin{itemize}
    \item \texttt{window\_size} – Width of the median window. Must be an odd integer.
\end{itemize}

\textbf{Pros:}
\begin{itemize}
    \item Excellent for removing impulsive noise (spikes).
    \item Preserves edges better than moving average.
\end{itemize}

\textbf{Cons:}
\begin{itemize}
    \item Not ideal for reducing continuous Gaussian noise.
    \item Can distort small peaks if the window is too large.
\end{itemize}

\textbf{Recommended Use:}  
Use when data contains discrete spike artifacts or experimental glitches.

\textbf{Avoid If:}  
Your data has small-amplitude features that could be suppressed by the median operation.

\subsection{Fourier Low-Pass Filter}

This method filters the signal in the frequency domain by removing high-frequency components above a specified cutoff. It is suitable for uniformly spaced data.

\textbf{Parameters:}
\begin{itemize}
    \item \texttt{cutoff} – Cutoff frequency (in units of inverse x-axis units), above which components are removed.
\end{itemize}

\textbf{Pros:}
\begin{itemize}
    \item Effective for removing high-frequency noise.
    \item Retains overall spectral shape well.
\end{itemize}

\textbf{Cons:}
\begin{itemize}
    \item Requires uniformly spaced data.
    \item Can introduce ringing artifacts (Gibbs phenomenon).
\end{itemize}

\textbf{Recommended Use:}  
Effective when noise is dominated by high-frequency components and the data is evenly spaced.

\textbf{Avoid If:}  
The data is unevenly sampled, or if phase preservation is critical (e.g., for derivative spectroscopy).

\section{Batch analysis}
Batch analysis offers an automated way of analyzing all the spectra in the selected folder. For using this functionality, just load one spectrum, select the desired range of analysis, and build/load the appropriate model. 

Following the execution of the Batch analysis, the application simply requests the format of the spectrum files it will process and the directory where they are located. It then proceeds to analyze each file in the folder, using the same boundaries as the spectrum used to build the model, optimizing the model parameters, and storing the images in a subfolder within the chosen directory. Once all spectra have been fitted, the optimized parameter values and their errors are documented in an Excel file within the same subfolder.

\section{1D mapping}

One-dimensional (1D) Raman mapping refers to the analysis of spectral evolution as a function of a single varying parameter. In the context of this application, 1D maps are typically acquired by recording Raman spectra sequentially while changing an external variable, such as electrochemical \textbf{potential}, elapsed \textbf{time}, or \textbf{position} along a linear spatial scan.

This type of analysis enables users to monitor dynamic changes in materials, detect chemical or structural gradients, and track reaction progress over time or space. The app supports the import, visualization, and automated processing of 1D datasets in a user-friendly format designed to allow efficient exploration of such trends.

Currently, this app is capable of importing only 1D files, specifically those from Horiba .txt files that have been exported as "split" arrays. To export the timemap and line scans in this format, navigate to the save button in LabSpec6 software, click the small arrow next to it, and select the "Split array" option. This will automatically export the map into separate files, including time or distance information in both the filename and the file header. For potential maps (SEC - SpectroElectroChemistry), the files must include the potential values applied to the working electrode during spectrum acquisition, with mV as the locator string for reading of the potential. For example, a filename such as \textit{$sample\_12\_-800mV.txt$} indicates a spectrum recorded at -800mV.

The loading of the Witec format will be added in the future.

To load the dataset, first specify the source spectroscope (Horiba or Witec) along with the measurement mode to be analyzed (timemap, linemap, SEC). Clicking the "Load" button will then prompt a window where you can select the directory containing the .txt files.

Another section allows the user to specify the map's display details. The left and right cutoffs set the spectral range to be plotted, while the lower and upper indices define the index region (such as time, length, or potential). The colorscale indicates the color scheme for data visualization. The correction option provides basic background correction for data display: "none" for raw data, "zero" sets the lowest point of the spectrum to zero, and "linear" calculates and subtracts a line profile from the raw data using the first and last points of each spectrum. Orientation specifies whether data is shown from top to bottom or the reverse. Interpolation offers graphical smoothing for the image, applicable only to heatmaps, with various strengths available (these are built-in functions of the `imshow` function in matplotlib; further details are in the matplotlib documentation).

\subsection{Heatmap}
The heatmap presentation of the 1D map data is based on the restructuring of the input data in the 2D matrix using the Raman shift and the index as columns and index, respectively. The intensity values are then used as the variable for the color coding of the data set. The setting of the smoothing can help with the lower the spectral noise and steps in the picture caused by small index sampling. To mark the desired area, right-click on the menu and select the "Select region" option. You can then click and drag the mouse over the image to highlight the area of interest. Upon releasing the mouse button, the spectral and index limits will be updated accordingly.

An additional capability provided by this heatmap environment is the option to plot the spectrum directly onto the Main window. To enable this feature, simply perform a right-click on the desired spectrum image that you wish to examine thoroughly, and then choose the "Plot in Main" option from the context menu.
\subsection{Lineplot}
Another way to present a 1D Raman map is through a line plot. This method displays the spectra as a line plot with an offset determined by the slider. The slider adjusts the offset based on a percentage of the dataset's maximum intensity, making it dependent on the correction settings. The spectra are subsequently color-coded according to their indices. In this case, by right-clicking and choosing "Select region," only the spectral range will be selected. The exclusion of the indexes must be performed manually.
\subsection{Fitting of the 1D map}
The Fit button triggers the fitting process for the one-dimensional map across the specified spectral range using the composite model set in the main window. In case a model is not yet specified, you may export the spectrum from the heatmap to the main window, where you can either manually define the model or import it from an existing .json file. Once the fitting process concludes, the resulting fit parameters and their corresponding indices are saved into an Excel file.
\section{2D mapping}
2D Raman mapping is a procedure in which Raman spectra are measured at different points on a surface, creating a two-dimensional map. Each pixel at this map represents a Raman spectrum, providing detailed information about the sample at each location.
Currently this app can analyses only the maps from Witec spectrometer, exported as table with header where columns contains the information about the pixel position (automatic export in Project5/6).

Loading of the Horiba 2D maps will be added in the future.

In the upper segment of the right column, just beneath the loading button, users can specify the map parameters. This includes defining the spectral region intended for analysis, along with setting the x and y range, allowing for analysis restricted to specific part of the map. The spectral region requires manual setting; however, the x and y limits can be set by right-clicking on the map, selecting the "Select region" option, and then using a left-click to designate the desired area on the map. This function becomes particularly advantageous when a map is already being displayed.

\subsection{Metric evaluation}
The initial assessment of the map involves computing the pixel metric. At this point, users have the option to choose among three metrics: the area (which refers to the integration of the spectrum area within a chosen spectral region), the maximum value (indicating the highest data value found within that spectral region), and max\_position (which denotes the spectral location of this maximum value). To display the metric map, you need to click on the "Plot metric" button. Additionally, there are other parameters you can adjust: the "Colormap," which specifies the color palette for the map, and "Interpolation," similar to its application in a 1D heatmap. You have the option to focus on a specific region of the evaluated parameter by setting the Scale's minimum and maximum values. This approach can help highlight particular data trends. Any values exceeding the set maximum or falling below the minimum will appear as saturated pixels, with their colors corresponding to either the top or bottom of the chosen color scale. Once the metric analysis is done, it do not need to be run again unless the spectral region is changed. If the spectral range is changed the program will automatically detect it and run the new metric analysis after the next "Plot metric" click.

\subsection{Fitting of the 2D map}
To fit the 2D map, it is essential to have the model loaded in the main window, which will be used for fitting. There are multiple approaches to achieve this. If you're analyzing a known spectral features, you might already have a .json file containing the model from previous work. In that situation, simply load the .json model into the main window. If the model isn' available, you can manually build it in the main window. You can export the spectrum from a single pixel or the map's average by right-clicking on the map and choosing "Plot spectra in Main" or "Plot average in Main." It is necesseary to create the metric map first, to have some data to export.

Once the model setup is complete, pressing the "Fit" button initiates the fitting process. When it is finished, the options for plotting the fit variables become accessible. Simply select the desired variable to plot from the "Fit variable" dropdown menu, which is pre-filled with variables from the model used, and then click on the "Plot fit" button. The other configurations of the map are the same as those for the Metric map.

The fitted values can be saved as and excel file for later use using the button "Save fit".

\section{Excel plotter}
The primary function of this feature is to swiftly allow for the visualization of mapping outcomes. Once the Excel file is loaded, users can effortlessly choose the data sources for both the x and y axes, assign appropriate labels, and proceed to create the plot.